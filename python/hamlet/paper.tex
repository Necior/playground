\documentclass[12pt, a4paper, twocolumn]{article}
\usepackage{abstract}
\usepackage{amsmath}
\usepackage{amsfonts}
\newcommand{\hamlet}{\textit{hamlet}}
\title{\hamlet: a simplified model of the Tribalwars game}
\author{Adrian Sadłocha}
\date{????-??-??}

\newcommand{\abstractText}{
Tribalwars is an online browser game in which players build their own villages.
In this paper I propose a simplified graph model of the game, which I call \hamlet.
Our goal is to find "the best" gameplay, or the shortest path between an initial state and a target state.

An heuristic-based graph traversal algorithm is proposed, implemented and evaluated.

Finally, I present the results of using a \hamlet gameplay in real Tribalwars game.
}

\begin{document}

\twocolumn[
  \begin{@twocolumnfalse}
    \maketitle
    \begin{abstract}
      \abstractText
      \newline
      \newline
    \end{abstract}
  \end{@twocolumnfalse}
]

\section{Initial model}

The Tribalwars game has a complex set of rules.
Some of the rules differ between localized versions of the game or between worlds.

In this section I propose a simplified model of the game.
The main idea is to use graph vertices to represent game states and use (directed) edges to represent player actions.
This way we can treat graph paths as possible gameplays.

\subsection{Game model}

For starters, let's consider a simplified model of the game.

Let:

\begin{equation*}
  \begin{aligned}
    G &= (t, R, L)\\
      &= (t, (R_w, R_c, R_i), (tc, cp, im)),
  \end{aligned}
\end{equation*}

where:

\begin{itemize}
\item \(G\) - the game state,
\item \(t\) - time since the beginning of the game,
\item \(R\) - available resources,
\item \(L\) - building levels,
\item \(R_w\) - available resources: wood,
\item \(R_c\) - available resources: clay,
\item \(R_i\) - available resources: iron,
\item \(tc\) - level of a timber camp,
\item \(cp\) - level of a clay pit,
\item \(im\) - level of an iron mine.
\end{itemize}

\(R_w, R_c, R_i, tc, cp, im\) are non-negative integers.
Moreover, we've got the following constraints on the building levels:

\begin{equation*}
\begin{aligned}
  1 &\leq tc \leq 30 &= TC \\
  1 &\leq cp \leq 30 &= CP \\
  1 &\leq im \leq 30 &= IM.
\end{aligned}
\end{equation*}

\subsection{Game state}

The most prevalent game state is the initial state \(S = (0, (0, 0, 0), (1, 1, 1))\),
i.e.\ a state in which there are no available resources and all buildings are at their minimum levels.

\subsection{Gameplay}

Gameplay is a finite sequence of transitions which starts in the initial state.

A \textit{successful} gameplay finishes in the target state \(T = (t, R, (TC, CP, IM))\),
i.e.\ a state in which all the buildings are at the maximum level.

The most desired gameplays are those which are successful and minimize gamplay time \(t\).

\subsection{Transition}

A missing definition is a transition definition.

In order to achieve the target state, we have to perform a (finite, unknown) number of transitions.

Informally, a transition in our game model corresponds to an action in the game.

For example, we can upgrade our timber camp from level 1 to level 2.
Such a transition is an edge between game states \(G_1\) and \(G_2\).

\subsubsection{Possible transitions}
In general, there are 4 possible transitions in \hamlet:

\begin{itemize}
  \item wait \(n \in \mathbb{N}^+\) hours,
  \item upgrade a timber camp,
  \item upgrade a clay pit,
  \item upgrade an iron mine.
\end{itemize}

Waiting is a way to get more resources.
An exact amount of new resources depends on building levels and waiting time.

Upgrading changes a building level from \(i\) to \(i + 1\).
Each upgrade has a specified cost in resources.
When a building reaches its maximum level (e.g. \(TC\) for a timber camp), it can no longer be upgraded.

% TODO: define `next(G, T)`, define `cost(T)`.
% TODO: provide an example (upgrading timber camp from level 1 to level 2 costs (63, 77, 50).

\end{document}